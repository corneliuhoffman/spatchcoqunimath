\documentclass[11pt, oneside]{article}   	% use "amsart" instead of "article" for AMSLaTeX format
\usepackage{geometry}                		% See geometry.pdf to learn the layout options. There are lots.
\geometry{letterpaper}                   		% ... or a4paper or a5paper or ... 
%\geometry{landscape}                		% Activate for for rotated page geometry
%\usepackage[parfill]{parskip}    		% Activate to begin paragraphs with an empty line rather than an indent
\usepackage{graphicx}				% Use pdf, png, jpg, or eps§ with pdflatex; use eps in DVI mode
								% TeX will automatically convert eps --> pdf in pdflatex		
\usepackage{amssymb}

\title{Starting test}
\author{}
%\date{}							% Activate to display a given date or no date

\begin{document}
\maketitle
%\section{}
%\subsection{}
\begin{enumerate}


\item Prove that if m + n and n + p are even integers, where m, n, and p are integers, then m + p is even. 


\item Show that every odd integer is the difference of two squares.

\item Use a proof by cases to show that $\min(a, \min(b, c)) = \min(\min(a, b), c)$ whenever a, b, and c are real numbers.



\item The Logic Problem, taken from WFF?N PROOF, The Game of Logic, has these two assumptions:

1. Logic is difficult or not many students like logic.

2. If mathematics is easy, then logic is not difficult.

 By translating these assumptions into statements involving propositional variables and logical connectives, determine whether each of the following are valid conclusions of these assumptions:

a) That mathematics is not easy, if many students like logic.
 
b) That not many students like logic, if mathematics is not easy.
 
c) That mathematics is not easy or logic is difficult.
 
d) That logic is not difficult or mathematics is not easy.

e) That if not many students like logic, then either mathematics is not easy or logic is not difficult.
\end{enumerate}



\end{document}  