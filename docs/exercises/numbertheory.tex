\documentclass[11pt, oneside]{article}   	% use "amsart" instead of "article" for AMSLaTeX format
\usepackage{geometry}                		% See geometry.pdf to learn the layout options. There are lots.
\geometry{letterpaper}                   		% ... or a4paper or a5paper or ... 
%\geometry{landscape}                		% Activate for for rotated page geometry
%\usepackage[parfill]{parskip}    		% Activate to begin paragraphs with an empty line rather than an indent
\usepackage{graphicx}				% Use pdf, png, jpg, or eps§ with pdflatex; use eps in DVI mode
\newtheorem{theorem}{Theorem}							% TeX will automatically convert eps --> pdf in pdflatex		
\usepackage{amssymb}
\usepackage{pmboxdraw}

\usepackage{fontspec}
\setmainfont{FreeSerif}
\setmonofont{FreeMono}
\title{First number theory set}
\author{}
\date{}							% Activate to display a given date or no date

\begin{document}
\maketitle
%\section{}
%\subsection{}

\newtheorem{axiom}{Axiom}
\newtheorem{lemma}{Lemma}
\newtheorem{definition}{Definition}

\newtheorem{puzzle}{Puzzle}

Prove the Lemmas bellow. Name them as before as oneone,  twoone and so on. Please make sure you save the work as you go along and submit as many solutions as you want. I would prefer 10  flawed solutions to one perfect one. I want to see how you think about it, when do you guess, what searches you do, when do you make mistakes, how do you correct them and so on.
Note that you will need to start with
\begin{verbatim}
Definition div a b:= exists c, b = a* c.
Notation "a | b" := (div a b)(at level 0).
\end{verbatim}

You will also get to use the tactics:
\begin{verbatim}
Rewrite hypothesis VAR using the definition of VAR.
Rewrite goal using the definition of VAR.
\end{verbatim}


Another useful trick is to make use of the  command SearchPattern. For example, the following

\begin{verbatim}
SearchPattern (_ * (_ + _) = _).
\end{verbatim}
will find all usable theorems that look like that, it will respond with the message

\begin{verbatim}
Query commands should not be inserted in
scripts

Nat.mul_add_distr_l: ? n m p : nat, n * (m + p) = n * m + n * p
\end{verbatim}
 And therefore you can use the theorem Nat.mul\_add\_distr\_l in your work, either with 
 \begin{verbatim}
  Apply result Nat.mul_add_distr_l.
 \end{verbatim}
  or with
  
  \begin{verbatim}
Rewrite the goal using  Nat.mul_add_distr_l.
 \end{verbatim}

Another example would be
\begin{verbatim}
SearchPattern (_ * _ = 0 -> _).
\end{verbatim}
which gives:

\begin{verbatim}
Query commands should not be inserted in
scripts

mult_is_O: ? n m : nat, n * m = 0 ? n = 0 ? m = 0

Nat.eq_mul_0_l: ? n m : nat, n * m = 0 ? m ? 0 ? n = 0

Nat.eq_mul_0_r: ? n m : nat, n * m = 0 ? n ? 0 ? m = 0
========================
Nat.mul_eq_0_l: ? n m : nat, n * m = 0 ? m ? 0 ? n = 0
========================
Nat.mul_eq_0_r: ? n m : nat, n * m = 0 ? n ? 0 ? m = 0
\end{verbatim}


\begin{lemma} Let $a,b,c \in \mathbb{N}$
\begin{enumerate}

\item  if $a | b$ and $b | c$, then $a | c$.
\item If $a | c$ and $b | d$, then $ab | cd$.
\item If   $a | b $ and $a | c$  then $a | b+c$.
\item If $a \ne  0$ and $c \ne 0$ $ac | bc$, then $a | b$.
\end{enumerate}

\end{lemma}

\begin{lemma}
\begin{enumerate}

\item The sum of two odd numbers is even.
\item the sum of two consecutive numbers is odd
\item the product of two consecutive numbers is even.


\end{enumerate}

\end{lemma}


\begin{lemma}

\begin{enumerate} 
\item $\forall n \in \mathbb{N},  2 | (n * n + n)$.
\item $\forall n \in \mathbb{N}$, the sum of the first n numbers equals $n(n-1)/2$.
\item $\forall n \in \mathbb{N}$, he sum of the first n odd numbers equals $n^2$.
\item $\forall n \in \mathbb{N}, 2 | (3^n - 1)$.



\end{enumerate}

\end{lemma}

\end{document}  

 