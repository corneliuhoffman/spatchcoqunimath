
\documentclass[11pt, oneside]{article}   	

\usepackage{color}
\usepackage{graphicx}				
\usepackage{prfblock}
\usepackage{amssymb}
\newtheorem{Theorem}{Theorem}
\newtheorem{Lemma}{Lemma}
\newtheorem{Proposition}{Proposition}
\newtheorem{Definition}{Definition}
\newtheorem{Inductive}{Inductive Definition}
\newtheorem{Variable}{Variable}
\newtheorem{Notation}{Notation}
\newtheorem{Axiom}{Axiom}
 \usepackage{tcolorbox}
 \tcbuselibrary{skins}
 \tcbuselibrary{theorems}
\tcbuselibrary{breakable}


\newcommand{\mybox}[1]{\begin{tcolorbox}[colback=white,colframe=gray!20!white, breakable, skin=enhancedmiddle]#1 \end{tcolorbox}}
\title{Brief Article}
\author{The Author}
\date{}							% Activate to display a given date or no date

\begin{document}
\maketitle

\begin{Lemma}[a] \label{Lemma:a}
$a(P\,Q:Prop):P\lor Q\Rightarrow Q\lor P.$
 \end{Lemma}


 Proof: We will assume $$Hyp : P \lor Q $$ and show $$Q \lor P $$.Since we know $Hyp : P \lor Q $ we can consider two cases: 

 Case 1 $$Hyp0 : P $$

 {\color{red}I am stuck and cannot prove this!!}

 

 Case 2 $$Hyp1 : Q $$

 We will prove the left hand side of $Q \lor P $. That is we need to prove $$Q $$.{\color{red} THIS STILL NEEDS A PROOF}

 We have now proved $$Q $$ and so $Q \lor P $ follows.

 Since we proved both cases, we are now done with $Q \lor P $.

 We have now showed that if $$Hyp : P \lor Q $$ then $$Q \lor P $$ a proof of $(P \lor Q) \Rightarrow (Q \lor P) $.\end{document}
